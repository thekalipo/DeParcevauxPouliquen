This document has been prepared to help you approaching Latex as a formatting tool for your Travlendar+ deliverables. This document suggests you a possible style and format for your deliverables and contains information about basic formatting commands in Latex. A good guide to Latex is available here \href{https://tobi.oetiker.ch/lshort/lshort.pdf}{https://tobi.oetiker.ch/lshort/lshort.pdf}, but you can find many other good references on the web. 

okk

Writing in Latex means writing textual files having a \texttt{.tex} extension and exploiting the Latex markup commands for formatting purposes. Your files then need to be compiled using the Latex compiler. Similarly to programming languages, you can find many editors that help you writing and compiling your latex code. Here \href{https://beebom.com/best-latex-editors/}{https://beebom.com/best-latex-editors/} you have a short oviewview of some of them. Feel free to choose the one you like.  

Include a subsection for each of the following items\footnote{By the way, what follows is the structure of an itemized list in Latex.}:
\begin{itemize}
\item
Purpose: here we include the goals of the project
\item
Scope: here we include an analysis of the world and of the shared phenomena
\item
Definitions, Acronyms, Abbreviations
\item
Revision history
\item
Reference Documents 
\item
Document Structure
\end{itemize}
Below you see how to define the header for a subsection.
\subsection{Purpose}
First of all lets define the different goals.

\begin{itemize}
	\item 
	G2 : Allow the farmers to retrieve personalized suggestions (On what are based suggestions ? Weather, location, soil humidity, crop, best practices given by others)
	\item
	G3 : Allow the farmers to access data about weather and soil conditions and predictions (In what area ? 10/150 km What kind of data exactly ?)
	\item
	G5 : Allow farmers to discuss on forums (Check if it is part of G6 or if the means are separated)
	\item
	G.1 : Allow farmers to get advices for optimizing their production
	\subitem
	G.1.1 : Allow farmers to retrieve personalized suggestions if they perform poorly
	\subitem
	G.1.2 : Allow farmers to discuss with other farmers about their issues
	\subitem
	G1.2.1 : Allow farmers to create a discussion forum
	\subitem
	G1.2.2 : Allow farmers to look for a specific topic among discussion forums
	\subitem
	G1.2.3 : Allow farmers to send messages on a forum already created
	\subitem
	G1.2.4 : Allow farmers to contact another farmer privately	
	Allow farmers to send a specific help request
	\item
	G2 : Allow farmers to get data that impact their production. 
	Allow farmers to access data about weather conditions and predictions
	\subitem
	G2.1 :
	Allow farmers to access data about soil moisture
	Allow farmers to access data about soil organic carbon
	\item
	G3 : Allow policy makers to globally enhance the productivity of the farmers of their area
	
	Allow policy makers to identify well and poorly performing farmers of their area, according to a chosen metric
	Allow policy makers to incent well performing farmers
	Allow policy makers to fetch best practices among farmers and provide them to others
	Allow policy makers to send personalized suggestions to poorly performing farmers
	
\end{itemize}

\subsection{Scope}

\subsubsection{World Phenomena}
%what you write here is a comment that is not shown in the final text
\begin{itemize}
	\item
	Farmers seed their crops
	\item
	Farmers fertilize their crops
	\item
	Farmers measure the amount of fertilizer used for a specific cropping
	\item
	Farmers harvest their crops
	\item
	Farmers measure their production
	\item
	Water consumption figures are updated (every ... ? by farmers or another state system ?)
	\item
	Soil moisture figures are updated
	\item
	Vegetation index figures are updated
	\item
	Rainfall conditions are updated
	\item
	Rainfall previsions are updated
	\item
	Global Positioning System (GPS) gets the farmer location
\end{itemize}

\subsubsection{Shared Phenomena}
\begin{itemize}
	
	\item
	Farmer releases production data
	\item
	Farmer enters the fertilizers he uses
	\item
	Farmer releases the amount of fertilizer used for cropping
	\item
	Farmer receives special incentive
	\item
	Farmer receives a request of best practices
	\item
	Farmer provides best practices
	\item
	Farmer visualizes the weather forecasts
	\item
	Farmer visualizes personalized suggestions concerning crops and fertilizers
	\item
	Farmer requests for help
	\item
	Farmer creates discussion forum
	\item
	Farmer searches for a discussion forum on a specific topic
	\item
	Farmer sends a message in a discussion forum
	\item
	Farmer sends a message to another farmer
	\item
	Farmer registers and provides personal data (mail, name)
	\item
	Farmer logs in
	\item
	Farmer provides exploitation data (location, type of production)
	\item
	Policy Maker registers and provides personal data (mail, name)
	\item
	Policy Maker logs in
	\item
	Policy Maker provides area he is responsible of
	\item
	Policy Maker asks for a ranking of the farmers he is in charge of, based on some metric
	\item
	Policy Maker searchs for a discussion forum on a specific topic
	\item
	Policy Maker sends a message to a farmer
	\item
	DREAM displays a notification to farmer for production release
	\item
	DREAM displays a notification to farmer for lack of soil or weather data
	\item
	DREAM displays a notification to farmer for new message from another farmer
	\item
	DREAM displays a notification to farmer for new message from a policy maker
	\item
	DREAM displays a notification to farmer for help request proposal
	\item
	DREAM displays a notification to farmer for best practice
	\item
	DREAM displays a notification to farmer for e-voucher
	\item
	DREAM displays a notification to policy maker for help request from a farmer
	\item
	DREAM displays a notification to policy maker for new registration
	\item
	DREAM displays a notification to policy maker for completion of production data
	
\end{itemize}

\subsubsection{Machine Phenomena}
\begin{itemize}
	\item 
	Identifies best performing farmers with regard to meteorological events
	\item 
	Identifies worst performing farmers with regard to meteorological events
	\item 
	Compute personalized suggestions concerning crops and fertilizers
	\item 
	Fetchs weather forecasts
	\item 
	Fetchs data from water irrigation system (whatever it is)
	\item 
	Fetchs user's location
	
\end{itemize}