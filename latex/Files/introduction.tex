\subsection{Purpose}
As reported by the Food and Agriculture Organization of the United Nations \cite{fao}, the Indian cultivated area went through a great expansion between 1970 and 2000, going from 140 million to 190 million hectares (ha). In the meantime, the number of farmholding skyrocketed, from 70.5 to 115.6 million. This is the reason why, the average size of farmholdings dropped from 2.30 to 1.41 ha. 

Hence, the Indian government has to tackle many new farms with lower sizes, which may require some help or incentives.

In this contect, the United Nations Development Programme (UNDP) partnered with the state of Telangana to enhance good deviances in the state's agriculture through a data-driven approach. 

\begin{itemize}
	\item
	G.1 : Allow farmers to get advices for optimizing their production
	\subitem
	G.1.1 : Allow farmers to retrieve personalized suggestions if they perform poorly
	\subitem
	G.1.2 : Allow farmers to discuss with other farmers about their issues
	\subitem
	G1.2.1 : Allow farmers to create a discussion forum
	\subitem
	G1.2.2 : Allow farmers to look for a specific topic among discussion forums
	\subitem
	G1.2.3 : Allow farmers to send messages on a forum already created
	\subitem
	G1.2.4 : Allow farmers to contact another farmer privately	
	Allow farmers to send a specific help request
	\item
	G2 : Allow farmers to get data that impact their production. 
	\subitem
	G2.1 :Allow farmers to access data about weather conditions and predictions
	\subitem
	G2.2 :
	Allow farmers to access data about soil moisture
	\subitem
	G2.3
	Allow farmers to access data about soil organic carbon
	\item
	G3 : Allow policy makers to globally enhance the productivity of the farmers of their area
	\subitem
	G3.1 : Allow policy makers to identify well and poorly performing farmers of their area, according to a chosen metric
	\subitem
	G3.2 : Allow policy makers to incent well performing farmers
	\subitem
	G3.3 : Allow policy makers to fetch best practices among farmers and provide them to others
	\subitem
	G3.4 : Allow policy makers to send personalized suggestions to poorly performing farmers

	
\end{itemize}

\subsection{Scope}

\subsubsection{World Phenomena}
%what you write here is a comment that is not shown in the final text
\begin{itemize}
	\item
	Farmers seed their crops
	\item
	Farmers fertilize their crops
	\item
	Farmers measure the amount of fertilizer used for a specific cropping
	\item
	Farmers harvest their crops
	\item
	Farmers measure their production
	\item
	Water consumption figures are updated 
	\item
	Soil moisture figures are updated
	\item
	Vegetation index figures are updated
	\item
	Rainfall conditions are updated
	\item
	Rainfall previsions are updated
	\item
	Global Positioning System (GPS) gets the farmer location
\end{itemize}

\subsubsection{Shared Phenomena}
\begin{itemize}
	
	\item
	Farmer releases production data
	\item
	Farmer enters the fertilizers used
	\item
	Farmer enters the seed variety used
	\item
	Farmer enters the seed rate of the crop
	\item
	Farmer releases the amount of fertilizer used for cropping
	\item
	Farmer enters the start and end dates
	\item
	Farmer receives special incentive
	\item
	Farmer receives a request of best practices
	\item
	Farmer provides best practices
	\item
	Farmer visualizes the weather forecasts
	\item
	Farmer requests for help
	\item
	Farmer creates discussion forum
	\item
	Farmer searches for a discussion forum on a specific topic
	\item
	Farmer sends a message in a discussion forum
	\item
	Farmer sends a message to another farmer
	\item
	Farmer registers and provides personal data (mail, name)
	\item
	Farmer logs in
	\item
	Farmer provides exploitation data (location, type of production)
	\item
	Policy Maker registers and provides personal data (mail, name)
	\item
	Policy Maker logs in
	\item
	Policy Maker provides area he is responsible of
	\item
	Policy Maker asks for a ranking of the farmers he is in charge of, based on some metric
	\item
	Policy Maker searchs for a discussion forum on a specific topic
	\item
	Policy Maker sends a message to a farmer
	\item
	DREAM displays a notification to farmer for production release
	\item
	DREAM displays a notification to farmer for lack of soil or weather data
	\item
	DREAM displays a notification to farmer for new message from another farmer
	\item
	DREAM displays a notification to farmer for new message from a policy maker
	\item
	DREAM displays a notification to farmer for a suggestion from a policy maker
	\item
	DREAM displays a notification to farmer for help request proposal
	\item
	DREAM displays a notification to farmer for best practice
	\item
	DREAM displays a notification to farmer for e-voucher
	\item
	DREAM displays a notification to policy maker for help request from a farmer
	\item
	DREAM displays a notification to policy maker for new registration in his/her area
	\item
	DREAM displays a notification to policy maker for completion of production data
	
\end{itemize}

\subsubsection{Machine Phenomena}
\begin{itemize}
	\item 
	Identifies best performing farmers with regard to meteorological events
	\item 
	Identifies worst performing farmers with regard to meteorological events
	\item 
	Compute personalized suggestions concerning crops and fertilizers
	\item 
	Fetchs weather forecasts
	\item 
	Fetchs data from water irrigation system
	\item 
	Fetchs user's location
	
\end{itemize}

\subsection{Definitions, Acronyms, Abbreviations}
\subsubsection{Definitions}
\begin{itemize}
	\item 
\end{itemize}

\subsubsection{Acronyms}
\begin{itemize}
	\item RASD: Requirement Analysis and Specification Document
	\item DREAM: Data-dRiven PrEdictive FArMing in
Telengana
	\item GPS: Global Positioning System
	\item FAO: Food and Agriculture Organization
	\item UNDP: United Nations Development Programme
	\item UML: United Modeling Language
	\item PM: Policy Maker
	\item SQL: Simple Query Language
	\item NICES: National Inforamtion System for Climate and Environment Studies
	\item API: Application Programming Interface
\end{itemize}

\subsubsection{Abbreviation}
\begin{itemize}
	\item Gn: $n^{th}$ Goal
	\item Rn: $n^{th}$ Requirement
\end{itemize}

\subsection{Reference Documents}
\bibliographystyle{plain} 
\bibliography{refs}
