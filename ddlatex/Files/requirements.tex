The main goal of this document is to better explain what we have defined in the RASD document. In the following list we will explain how the requirements map to the design elements that are defined in this document.

\begin{itemize}
	\item
	R1: The system should integrate a calendar that rythms the alternance of cropping season/release period/analysis period.
	\subitem Components : Calendar 
	\item
	R2: Two weeks before the end of the release period, DREAM should recall the farmer to confirm the batch of production (if the farmer didn't confirm it yet).
	\subitem Components : Calendar, Message, Mail and Release
	\item
	R3: Two weeks before the end of the analysis period, DREAM should recall the policy maker to identify well and poorly performing farmers (if not done).
	\subitem Components : Calendar, Performance, Messaging and E-voucher
	\item
	R4: Two weeks before the end of the analysis period, DREAM should recall the policy maker to send messages/incentives/help proposal to farmers, if some of the well or poorly performing ones weren't contacted yet.
	\subitem Components : Calendar, Performance, Messaging and E-voucher
	\item
	R5: DREAM should enable messaging between farmers and a farmer and a policy maker from the same area.
	\subitem Components : Messaging, Mail and Geolocalization
	\subitem Mockup : figure \ref{Fig:interface_messages_policy_maker}
	\item
	R6: When an area is given, DREAM should fetch the policy maker in charge of it.
	\subitem Components : Geolocalization
	\item
	R7: DREAM should enable any farmer to create a discussion forum, with a title, tags and filters.
	\subitem Component : Forum
	\subitem Mockup : figure \ref{Fig:interface_forum}
	\item
	R8: When requested, DREAM should provide discussion forums ordered by semantic proximity of a given topic.
	\subitem Components : Forum, Geolocalization 
	\item
	R9: DREAM should enable unlimited answers in the thread of a forum.
	\subitem Component : Forum
	\item
	R10: When a location is asked and the user has agreed so, DREAM should fetch the position thanks to the GPS functionality of a device.
	\subitem Component : Geolocalization
	\item
	R11: Given a date and a location, DREAM should fetch from the external databases the  closest and latest weather conditions.
	\subitem Component : Visualization with the externals API (Wheather, Soil, VI)
	\subitem Mockup : figure \ref{Fig:interface_meteo}
	\item
	R12: Given a date, a location and a number of days, DREAM should fetch from the external databases the  closest and most adequate (with respect to the number of days of prediction) weather predictions.
	\subitem Components : Visualization with the external Wheather API and Geolocalization
	\item
	R13: Given a date and a location, DREAM should fetch from the external databases the  closest and latest soil moisture data (that is, figures and locations).
	\subitem Components : Visualization with the external Soil API and Geolocalization
	\item
	R14: Given a date and a location, DREAM should fetch from the external databases the  closest and latest soil organic carbon data (that is, figures and locations).
	\subitem Components : Visualization with the external Soil API and Geolocalization
	\item
	R15: Given a date and a location, DREAM should fetch from the external databases the  closest and latest vegetation index data (that is, figures and locations).
	\subitem Components : Visualization with the external VI API and Geolocalization
	\item
	R16: Given a list of figures and locations, DREAM should display them on an interactive map, with a nice legend.
	\subitem Component : Visualization
	\subitem Mockup : figure \ref{Fig:interface_meteo}
	\item
	R17: On a move, the map displayed should be reactive. That is to say: it should query new values to fit in the new spatial frame.
	\subitem Component : Visualization
	\subitem Mockup : figure \ref{Fig:interface_meteo}
	\item
	R18: DREAM should have a clock synchronous with the Indian time zone and Gregorian calendar.
	\subitem Component : Calendar
	\item
	R19: Given a metric and a batch from a location, DREAM should fetch the required data at the date that correspond to the batch. The system should then correctly compute the performance.
	\subitem Component : Performance, Geolocalization, Production DB, Release, Calendar
	\item
	R20: The system should be able to rank a list of farmers during a given season with respect to a metric.
	\subitem Component : Performance
	\item
	R21: Given a two dates and a farmer identity, DREAM should fetch from the water irrigation system the total amount of water consumed during the time-lapse by the farmer.
	\subitem Component : Performance with the Irrigation API
	\item
	R22: If any necessary piece of information is missing at some point of a process, DREAMS should display an error message.
	\subitem a check will be done by all the components and an error will be prompted in the front end.
	\item
	R23: Given a user identity, DREAMS should be able to access all the personal data of this user.
	\subitem Component : Personal data DB, the application will only collect the necessary data
	\item
	R24: When a policy maker decides to make an incentive, DREAMS should create via an external API an e-voucher of the amount wanted. 
	\subitem Components : Messaging with the E-voucher Generator
	\subitem Mockup : figure \ref{Fig:interface_messages_policy_maker}
	\item
	R25: Given a farmer identity, DREAM should be capable of fetching the corresponding batch.
	\subitem Component : Release and Production DB
	
\end{itemize}